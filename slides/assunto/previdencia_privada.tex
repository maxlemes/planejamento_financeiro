\section{Previdência Privada}

\begin{frame}[t]\frametitle{PGBL - Plano Gerador de Benefício Livre}
  \textbf{Características do PGBL:}
  \begin{itemize}
    \item Indicado para quem \textbf{declara o IR no modelo completo}.
    \item Permite \textbf{dedução de até 12\% da renda bruta anual} no IR.
    \item Tributação ocorre sobre \textbf{o valor total} (contribuições + rendimentos) no resgate.
    \item Pode ser vantajoso para quem tem renda mais alta e deseja reduzir o imposto a pagar no presente.
  \end{itemize}
\end{frame}

\begin{frame}[t]\frametitle{VGBL - Vida Gerador de Benefício Livre}
  \textbf{Características do VGBL:}
  \begin{itemize}
    \item Indicado para quem \textbf{declara o IR no modelo simplificado} ou é \textbf{isento}.
    \item Não permite dedução das contribuições no IR.
    \item Tributação ocorre apenas sobre \textbf{os rendimentos} no resgate.
    \item Mais vantajoso para quem não pode ou não quer aproveitar o benefício fiscal do PGBL.
  \end{itemize}
\end{frame}

\begin{frame}[t]\frametitle{Tributação Regressiva na Previdência Privada}

  \vspace{0.5cm}
  \textbf{Tabela de Alíquotas:}

  \begin{table}[]
    \centering
    \begin{tabular}{|c|c|}
      \hline
      \textbf{Tempo de Investimento} & \textbf{Alíquota de IR} \\ \hline
      Até 2 anos                     & 35\%                    \\ \hline
      De 2 a 4 anos                  & 30\%                    \\ \hline
      De 4 a 6 anos                  & 25\%                    \\ \hline
      De 6 a 8 anos                  & 20\%                    \\ \hline
      De 8 a 10 anos                 & 15\%                    \\ \hline
      Acima de 10 anos               & 10\%                    \\ \hline
    \end{tabular}
  \end{table}
\end{frame}

\begin{frame}[t]\frametitle{Exemplo: PGBL e Benefício Fiscal}
  \textbf{Situação Inicial:}
  \begin{itemize}
    \item Base de cálculo do IR: \textbf{R\$ 100.000}
    \item Alíquota marginal de IR: \textbf{27,5\%}
  \end{itemize}

  \vspace{0.3cm}
  \textbf{Após o Aporte no PGBL:}
  \begin{itemize}
    \item Aporte: \textbf{R\$ 10.000}
    \item Nova base de cálculo: \textbf{R\$ 90.000}
    \item \textbf{Economia de imposto: R\$ 2.750}
  \end{itemize}
\end{frame}
