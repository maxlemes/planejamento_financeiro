\section{Moradia: Financiamento Imobiliário vs. Aluguel}

\begin{frame}[c]\frametitle{Pagar aluguel do imóvel ou pagar aluguel do dinheiro?}
  \textbf{Vantagens de Alugar um Imóvel}
  \begin{itemize}
    \item \textbf{Flexibilidade:} Você pode mudar de casa ou cidade com facilidade.
    \item \textbf{Custo Inicial Baixo:} Não exige um valor alto de entrada.
    \item \textbf{Custos de Manutenção Reduzidos:} Você não é responsável por grandes reparos ou impostos do imóvel.
    \item \textbf{Valor menor no início:} O aluguel costuma ser menor que a parcela no início.
  \end{itemize}
\end{frame}

\begin{frame}[c]\frametitle{Financiamento Imobiliário vs. Aluguel}
  \textbf{Desvantagens de Alugar um Imóvel}
  \begin{itemize}
    \item \textbf{Não constrói patrimônio:} O dinheiro do aluguel não se converte em um ativo para você.
    \item \textbf{Incerteza e Instabilidade:} O contrato pode ser encerrado obrigando o inquilino a deixar o imóvel.
    \item \textbf{Falta de Autonomia:} Sem liberdade para fazer reformas ou personalizações no imóvel.
    \item \textbf{Valores crescentes:} O aluguel costuma ser reajustado anualmente com base na inflação.
  \end{itemize}
\end{frame}

\begin{frame}[c]\frametitle{Vantagens de se Financiar um Imóvel}
  \textbf{Vantagens do Financiamento:}
  \begin{itemize}
    \item \textbf{Criação de  Patrimônio:} Morar no que é seu.
    \item \textbf{Segurança e Estabilidade:} Livre das incertezas de contratos de aluguel.
    \item \textbf{Potencial de Valorização:} O imóvel se valoriza com o tempo.
    \item \textbf{Parcelas decrescentes:} O valor da parcela tende a diminuir com o tempo.
  \end{itemize}
\end{frame}

\begin{frame}[c]\frametitle{Desvantagens do Financiamento}
  \textbf{Os Riscos do Financiamento:}
  \begin{itemize}
    \item \textbf{Alto Custo:} O valor total: Saldo + Juros supera o valor do imóvel.
    \item \textbf{Custos Adicionais:} Seguro, taxa de administração e correção do saldo devedor.
    \item \textbf{Risco de Perda do Imóvel:} Em caso de inadimplência perde tudo o que pagou.
    \item \textbf{Risco de Inviabilidade:} O saldo devedor pode superar o valor do imóvel ao longo do financiamento.
  \end{itemize}
\end{frame}

\begin{frame}[c]\frametitle{Quando o Financiamento vale a pena}
  \textbf{Analisando um Financiamento:}
  \begin{itemize}
    \item \textbf{Taxa de juros baixa:} Os juros pagos devem ser no máximo o valor do aluguel.
    \item \textbf{Correção do Saldo Devedor:} Fuja de IPCA ou IGPM, prefira TR.
    \item \textbf{Pagar em dia:} Em hipótese alguma fica inadimplente.
    \item \textbf{Pagar o mais rápido que puder:} Tente antecipar as parcelas.
  \end{itemize}
\end{frame}

\begin{frame}[c]\frametitle{Simulação SACRE: Premissas}
  \textbf{Premissas do Cálculo}
  \begin{itemize}
    \item \textbf{Valor do imóvel:} R\$450.000,00
    \item \textbf{Entrada:} R\$90.000,00 (20\%)
    \item \textbf{Valor Financiado:} R\$360.000,00
    \item \textbf{Prazo:} 30 anos (360 meses)
    \item \textbf{Taxa de Juros:} 10\% a.a. ($\approx$ 0,7974\% ao mês)
    \item \textbf{Taxa de Administração:} 1\% a.a. ($\approx$ 0,0829\% ao mês)
  \end{itemize}
\end{frame}

\begin{frame}[c]\frametitle{Simulação SACRE: Com juros de 10 \% a.a.}
  \textbf{Cálculo da Parcela:} Amortização + Juros + Taxa de Administração
  \begin{itemize}
    \item \textbf{Amortização:} R\$360.000,00 / 360 meses = R\$1.000,00
    \item \textbf{Juros (1º mês):} R\$360.000,00 x 0,7974\% = R\$2.870,64
    \item \textbf{Taxa Adm. (1º mês):} R\$360.000,00 x 0,0829\% = R\$298,44
    \item \textbf{Prestação 1º ano:} R\$1.000,00 + R\$2.870,64 + R\$298,44 = \textbf{R\$4.169,08}
  \end{itemize}


  \begin{itemize}
    \item A parcela de \textbf{R\$4.169,08} ficarará fixa nos próximos 12 meses.
  \end{itemize}
\end{frame}

\begin{frame}[c]\frametitle{Simulação SACRE: Com juros de 10 \% a.a.}
  \begin{itemize}
    \item {Despesas do Financiamento:} \textbf{R\$3.169,08}
    \item {Aluguel Estimado:} \textbf{R\$2.500,00}
    \item {80\% do Aluguel Estimado:} \textbf{R\$2.000,00}
    \item {Diferença (mensal):} \textbf{R\$1.169,08} (+ 58,45\%)
  \end{itemize}
\end{frame}

\begin{frame}[c]\frametitle{Simulação SACRE: Com juros de 5\% a.a.}
  \textbf{Cálculo da Parcela:} Amortização + Juros + Taxa de Administração
  \begin{itemize}
    \item \textbf{Amortização:} R\$360.000,00 / 360 meses = R\$1.000,00
    \item \textbf{Juros (1º mês):} R\$360.000,00 x 0,4074\% = R\$1.466,64
    \item \textbf{Taxa Adm. (1º mês):} R\$360.000,00 x 0,0829\% = R\$298,44
    \item \textbf{Prestação 1º ano:} R\$1.000,00 + R\$1.466,64 + R\$298,44 = \textbf{R\$2.765,08}
  \end{itemize}


  \begin{itemize}
    \item A parcela de \textbf{R\$2.765,08} ficarará fixa nos próximos 12 meses.
  \end{itemize}
\end{frame}

\begin{frame}[c]\frametitle{Simulação SACRE: Com juros de 5 \% a.a.}
  \begin{itemize}
    \item {Despesas do Financiamento:} \textbf{R\$1.765,08}
    \item {Aluguel Estimado:} \textbf{R\$2.500,00}
    \item {80\% do Aluguel Estimado:} \textbf{R\$2.000,00}
    \item {Diferença (mensal):} \textbf{-R\$234,92} (-11,74\%)
  \end{itemize}
\end{frame}