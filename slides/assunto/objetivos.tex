\section{Objetivos do Planejamento Financeiro Pessoal}


\begin{frame}[t]\frametitle{Por que Planejar suas Finanças?}

  \vspace{1cm}\pause

  % --- Primeiro Item com Ícone ---
  \begin{block}{\Large \textcolor{green}{\faIcon{smile}}\quad Melhorar a qualidade de vida:}
    \begin{itemize}
      \item Aproveite ao máximo seu dinheiro, evitando gastos desnecessários.
    \end{itemize}
  \end{block}

  \vspace{0.7cm}\pause

  \begin{block}{\Large \textcolor{blue}{\faIcon{tasks}}\quad Assumir o controle:}
    \begin{itemize}
      \item Alinhe seus gastos aos seus objetivos.
    \end{itemize}
  \end{block}

  \vspace{0.7cm}\pause

  \begin{block}{\Large \textcolor{red}{\faIcon{bullseye}}\quad Construir o futuro:}
    \begin{itemize}
      \item Tenha a tranquilidade financeira necessária para concretizar seus objetivos.
    \end{itemize}
  \end{block}

\end{frame}

\section{Por onde começar?}

\begin{frame}[c]
  \frametitle{}

  % Bloco 1: O passo mental e de visão de futuro
  \begin{block}{\Large \textcolor{blue}{\faIcon{compass}}\quad Autoconhecimento (O Mapa)}

    \begin{itemize}
      \item O que é inegociável na sua vida? (Ex: segurança, conforto, educação) \pause
      \item O que significa "qualidade de vida" para você? \pause
      \item Quais são seus maiores objetivos de curto, médio e longo prazo? \pause
      \item Qual é a ordem de prioridade desses objetivos?
    \end{itemize}
  \end{block}
\end{frame}


\begin{frame}[c]
  \frametitle{}

  % Bloco 2: O passo prático e de diagnóstico
  \begin{block}{\Large \textcolor{red}{\faIcon{clipboard-list}}\quad Organização (O Raio-X)}

    \begin{itemize}
      \item Quais são suas fontes de renda mensal? (Salário, extras, etc.)
      \item Como você gasta seu dinheiro? Liste seus gastos fixos e variáveis.
      \item \textbf{Balanço:} Você gasta mais ou menos do que ganha?
      \item \textbf{Alinhamento:} Seus gastos atuais te aproximam ou te afastam dos seus objetivos?
    \end{itemize}
  \end{block}

\end{frame}


\begin{frame}[t]
  \frametitle{4 Vilões da Sua Qualidade de Vida}

  % --- Vilão 1 ---
  \begin{block}{\large \textcolor{orange}{\faIcon{sign-out-alt}}\quad O Ciclo do "Zera a Conta"}
    \textbf{Hábito:} Gastar 100\% da sua renda todo mês. \\
    \textbf{Consequência:} Viver sem nenhuma margem para erros ou oportunidades. Qualquer imprevisto se torna uma crise e o futuro nunca chega.
  \end{block}

  % --- Vilão 2 ---
  \begin{block}{\large \color{red}\faIcon{calendar-times}\quad A Prisão das Prestações}
    \textbf{Hábito:} Acumular parcelamentos e financiamentos em excesso. \\
    \textbf{Consequência:} Sua renda futura já está comprometida. Você perde a liberdade de escolha e trabalha para pagar o passado em vez de construir o futuro.
  \end{block}

  % --- Vilão 3 ---
  \begin{block}{\large \color{purple}\faIcon{exclamation-triangle}\quad A Corda Bamba Financeira}
    \textbf{Hábito:} Não ter uma reserva de emergência. \\
    \textbf{Consequência:} Viver em constante estresse e vulnerabilidade. Um problema de saúde ou a perda do emprego pode levar a dívidas impagáveis.
  \end{block}

  % --- Vilão 4 ---
  \begin{block}{\large \color{brown}\faIcon{credit-card}\quad O Impulso no Crédito}
    \textbf{Hábito:} Usar o cartão de crédito ou crediário para compras por impulso e bens de consumo. \\
    \textbf{Consequência:} Pagar juros por algo que perde valor rapidamente. Você paga mais caro por um prazer temporário, sacrificando objetivos maiores.
  \end{block}

\end{frame}