\section{Objetivos do Planejamento Financeiro Pessoal}


\begin{frame}[t]\frametitle{Por que Planejar suas Finanças?}

  \vspace{1cm}\pause

  % --- Primeiro Item com Ícone ---
  \begin{block}{\Large \textcolor{green}{\faIcon{smile}}\quad Melhorar a qualidade de vida:}
    \begin{itemize}
      \item Aproveite ao máximo seu dinheiro, evitando gastos desnecessários.
    \end{itemize}
  \end{block}

  \vspace{0.7cm}\pause

  \begin{block}{\Large \textcolor{blue}{\faIcon{tasks}}\quad Assumir o controle:}
    \begin{itemize}
      \item Alinhe seus gastos aos seus objetivos.
    \end{itemize}
  \end{block}

  \vspace{0.7cm}\pause

  \begin{block}{\Large \textcolor{red}{\faIcon{bullseye}}\quad Construir o futuro:}
    \begin{itemize}
      \item Tenha a tranquilidade financeira necessária para concretizar seus objetivos.
    \end{itemize}
  \end{block}
\end{frame}

\section{Por onde começar?}

\begin{frame}[c]
  \frametitle{}

  % Bloco 1: O passo mental e de visão de futuro
  \begin{block}{\Large \textcolor{blue}{\faIcon{compass}}\quad Autoconhecimento (O Mapa)}

    \begin{itemize}
      \item O que é inegociável na sua vida? (Ex: segurança, conforto, educação) \pause
      \item O que significa "qualidade de vida" para você? \pause
      \item Quais são seus maiores objetivos de curto, médio e longo prazo? \pause
      \item Qual é a ordem de prioridade desses objetivos?
    \end{itemize}
  \end{block}
\end{frame}


\begin{frame}[c]
  \frametitle{}

  % Bloco 2: O passo prático e de diagnóstico
  \begin{block}{\Large \textcolor{red}{\faIcon{clipboard-list}}\quad Organização (O Raio-X)}

    \begin{itemize}
      \item Quais são suas fontes de renda? (Salário, extras, etc.)\pause
      \item Como você gasta seu dinheiro? Liste seus gastos fixos e variáveis.\pause
      \item Comece pelos grandes blocos: Moradia, Educação, Transporte, Alimentação, Saúde, Lazer e Outros.\pause
      \item Olhe sempre o custo anual: existem muitos gastos sazonais, como IPTU, IPVA, seguros, matrículas, etc.
      \item \textbf{Balanço:} Você gasta mais ou menos do que ganha?\pause
      \item \textbf{Alinhamento:} Seus gastos atuais te aproximam ou te afastam dos seus objetivos?
    \end{itemize}
  \end{block}
\end{frame}

\begin{frame}[c]\frametitle{}

  \begin{block}{\Large \textcolor{blue}{\faIcon{lightbulb}}\quad Encontre sua Motivação:}

    \vspace{0.7cm}\pause

    \begin{itemize}
      \item O \textbf{dinheiro} Ele é apenas uma ferramenta para construir seus objetivos:
            \begin{itemize}
              \item Fazer aquela \textbf{viagem dos sonhos}.
              \item Comprar aquela \textbf{casa maravilhosa}.
              \item Garantir a \textbf{educação dos filhos}.
              \item Construir uma \textbf{aposentadoria tranquila}, onde você não dependa do governo ou de favores.
              \item Alcançar a \textbf{liberdade} de escolha, seja para abrir um negócio, trabalhar menos ou simplesmente mudar de ramo.
            \end{itemize}
    \end{itemize}
  \end{block}
\end{frame}

\begin{frame}[c]\frametitle{}
  \begin{block}{\Large \textcolor{green}{\faIcon{rocket}}\quad Monte seu Plano de Ação:}

    \vspace{0.7cm}\pause

    \begin{itemize}
      \item \textbf{Escolha sua ferramenta:} pode ser um aplicativo de finanças, uma planilha detalhada ou um simples caderno. O importante é que funcione para você. Registre \textit{tudo} o que entra e sai.
      \item \textbf{Crie seu Orçamento:} defina quanto você pretende gastar com despesas essenciais (moradia, alimentação), com  desejos (lazer, hobbies) e quanto destinará para metas financeiras (quitar dívidas, investir). Ajuste conforme sua realidade.
      \item \textbf{Automatize seus objetivos;} Configure transferências automáticas mensais da sua conta corrente para uma conta de investimentos ou poupança assim que seu salário cair. Isso torna a economia um hábito, não uma escolha.
    \end{itemize}
  \end{block}
\end{frame}

\begin{frame}[c]\frametitle{}
  \begin{block}{\Large \textcolor{green}{\faIcon{rocket}}\quad Monte seu Plano de Ação:}

    \vspace{0.7cm}

    \begin{itemize}
      \item \textbf{Estabeleça metas SMART:} Transforme sonhos em planos concretos. Exemplo: "Guardar R\$ 1.000 (Específico, Mensurável) para viajar para a Europa (Atingível, Relevante) daqui a  36 meses (Temporal)".
      \item \textbf{Faça revisões mensais:} Uma vez por mês, analise seus gastos e verifique o progresso em direção às suas metas. Celebre as vitórias e ajuste a rota se necessário.
    \end{itemize}
  \end{block}
\end{frame}

\section{Como fazer?}

\begin{frame}[t]
  \frametitle{Evite os inimigos da tranquilidade financeira}


  % --- Vilão 1 ---
  \begin{block}{\large \color{purple}\faIcon{exclamation-triangle}\quad A Corda Bamba Financeira}
    \begin{itemize}
      \item \textbf{Hábito:} Não ter uma reserva de emergência: "se algo acontecer, na hora eu resolvo".
      \item \textbf{Consequência:}  Pequenos imprevistos (saúde, carro, casa) se transformam em dívidas caras, e a ansiedade financeira tira a sua liberdade de escolha.
    \end{itemize}
  \end{block}

  \vspace{1cm}\pause

  % --- Vilão 2 ---
  \begin{block}{\large \color{orange}\faIcon{sign-out-alt}\quad O Ciclo do "Zera a Conta"}
    \begin{itemize}
      \item \textbf{Hábito:} Gastar 100\% da sua renda todo mês.
      \item \textbf{Consequência:} Viver sem nenhuma margem para erros ou oportunidades. Qualquer imprevisto se torna uma crise e o futuro nunca chega.
    \end{itemize}
  \end{block}
\end{frame}

\begin{frame}[c]\frametitle{}

  % --- Vilão 3 ---
  \begin{block}{\large \color{red}\faIcon{calendar-times}\quad A Prisão das Prestações}
    \begin{itemize}
      \item  \textbf{Hábito:} Acumular parcelamentos e financiamentos em excesso.
      \item \textbf{Consequência:} trabalhar para pagar o passado em vez de construir o futuro. perdendo a liberdade de escolha.
    \end{itemize}
  \end{block}

  \vspace{1cm}\pause

  % --- Vilão 4 ---
  \begin{block}{\large \color{brown}\faIcon{credit-card}\quad O Impulso no Crédito}
    \begin{itemize}
      \item \textbf{Hábito:} Usar o crédito para comprar bens de consumo.
      \item \textbf{Consequência:} Você paga mais caro por um prazer temporário que perde valor rapidamente, sacrificando objetivos maiores.
    \end{itemize}
  \end{block}
\end{frame}