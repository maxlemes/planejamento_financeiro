\section{Maus Hábitos Financeiros}

\begin{frame}[c]\frametitle{Comportamentos a Evitar}
  \begin{itemize}
    \item \textbf{Desorganização Financeira:} Não ter controle financeiro nem metas claras para o seu dinheiro.
    \item \textbf{Falta de Poupança:} Gastar tudo que se ganha. Isso impede a criação de patrimônio e a realização de objetivos financeiros de longo prazo.
    \item \textbf{Não ter uma reserva de emergência:} Viver sem um fundo de segurança para imprevistos.
    \item \textbf{Ignorar o Futuro:} Focar apenas no presente e não planejar objetivos de longo prazo.
    \item \textbf{Compras por Impulso:} Gastar dinheiro sem um planejamento prévio, quase sempre por uma satisfação momentânea.
          %\item \textbf{Investir sem Estratégia:} Tomar decisões de investimento sem conhecimento ou pesquisa. Isso pode levar a perdas financeiras, colocar o dinheiro em produtos inadequados para o seu perfil ou objetivo.
  \end{itemize}
\end{frame}

\section{Maus Hábitos Financeiros de Servidores Públicos}

\begin{frame}[c]\frametitle{Comportamentos mais comuns devido à estabilidade}
  \begin{itemize}
    \item \textbf{Uso Excessivo de Consignado:} Comprometer grande parte do salário com dívidas de longo prazo, reduzindo a capacidade de poupança.
    \item \textbf{Gastar Vantagens Temporárias:} Incorporar ao orçamento mensal valores de gratificações que podem ser perdidos, criando um desequilíbrio grande no futuro.
    \item \textbf{Procrastinar o Planejamento Financeiro:} A falsa sensação de segurança faz com que muitos adiem o planejamento financeiro, desconsiderando as mudanças nas regras de aposentadoria.
    \item \textbf{Zona de Conforto Financeiro:} A estabilidade do cargo público inibe a busca por conhecimento financeiro. Isso leva a decisões financeiras ruins, dependência de produtos bancários caros e dificuldade em aumentar o patrimônio.
  \end{itemize}
\end{frame}

