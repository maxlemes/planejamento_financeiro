\section{Previdência Pública}

\begin{frame}[c]\frametitle{Emenda Constitucional nº 20/1998 (Governo FHC)}
  \textbf{Principais Mudanças}
  \begin{itemize}
    \item Tempo mínimo de contribuição: 30 anos (mulheres) e 35 anos (homens).
    \item Idade mínima para servidores públicos: 55 (M) e 60 (H).
    \item Fim da aposentadoria proporcional para novos servidores.
    \item Exigência de tempo no serviço público e no cargo.
    \item Criação do fator previdenciário.
  \end{itemize}
\end{frame}

\begin{frame}[c]\frametitle{Emenda Constitucional nº 41/2003 (Governo Lula)}
  \textbf{Principais Mudanças}
  \begin{itemize}
    \item Criou a \textbf{contribuição para servidores aposentados} e pensionistas.
    \item Aposentados passaram  a pagar $11\%$  sobre o valor que excede o teto do RGPS.
    \item Extinguiu a \textbf{integralidade e paridade} para novos servidores.
    \item O salário base passou a ser a média das $80\%$ maiores remunerações.
  \end{itemize}
\end{frame}

\begin{frame}[c]\frametitle{Lei nº 12.618/2013 (Governo Dilma)}
  \textbf{Principal Mudança}
  \begin{itemize}
    \item Cria a regra 85(M)/95(H) (idade + tempo de contribuição).
    \item Limitou do benefício dos servidores ao teto do RGPS.
    \item Criação da Funpresp (previdência complementar para servidores federais).
    \item Servidor com salário acima do teto pode aderir à Funpresp para complementar a renda futura.
  \end{itemize}
\end{frame}

\begin{frame}[c]\frametitle{Emenda Constitucional nº 103/2019 (Governo Bolsonaro)}
  \textbf{Principais Mudanças}
  \begin{itemize}
    \item Idade mínima: 62 anos (mulheres) e 65 anos (homens)
    \item Tempo mínimo de 20 anos no serviço público e 5 anos no cargo.
    \item Novo cálculo do benefício: média de \textbf{100\%} das remunerações desde julho/1994.
    \item Fórmula: 60\% da média + 2\% por ano que exceder 20 anos de contribuição (homens) ou 15 anos (mulheres).
    \item Pensão por morte: 50\% do valor + 10\% por dependente, até 100\%.
    \item Contribuição previdenciária: alíquotas progressivas de 7,5\% a 22\%.
  \end{itemize}
\end{frame}