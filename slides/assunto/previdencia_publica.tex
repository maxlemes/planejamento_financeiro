\section{Previdência Pública}

\begin{frame}[c]\frametitle{Emenda Constitucional nº 20/1998 (Governo FHC)}
  \textbf{Principais Mudanças}
  \begin{itemize}
    \item Tempo mínimo: 30 anos (mulheres) e 35 anos (homens)
    \item Idade mínima para servidores públicos: 55 (M) e 60 (H)
    \item Fim da aposentadoria proporcional
    \item Criação do fator previdenciário
  \end{itemize}
\end{frame}

\begin{frame}[c]\frametitle{Emenda Constitucional nº 41/2003 (Governo Lula)}
  \textbf{Principais Mudanças}
  \begin{itemize}
    \item Criou a \textbf{contribuição para servidores aposentados}.
    \item Extinguiu a \textbf{integralidade e paridade} para novos servidores.
    \item Benefícios passaram a ser a média $80\%$ dos maiores salários de contribuição.
  \end{itemize}
\end{frame}

\begin{frame}[c]\frametitle{Lei nº 12.618/2013 (Governo Dilma)}
  \textbf{Principal Mudança}
  \begin{itemize}
    \item Limitação do benefício dos servidores ao teto do INSS
  \end{itemize}
\end{frame}

\begin{frame}[c]\frametitle{Emenda Constitucional nº 103/2019 (Governo Bolsonaro)}
  \textbf{Principais Mudanças}
  \begin{itemize}
    \item Idade mínima: 62 anos (mulheres) e 65 anos (homens)
    \item Novo cálculo do benefício: média de $100\%$ das contribuições
    \item Nova regra de cálculo: 60\% da média salarial + 2\% por ano extra.
    \item Mudanças na pensão por morte: 50\% + 10\% por dependente.
    \item Alíquotas progressivas de contribuição chegando a $22\%$
  \end{itemize}
\end{frame}