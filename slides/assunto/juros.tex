% !TEX root = main.tex
\section{Juros} \label{FormProb}


\begin{frame}[c]
    \frametitle{}
    \begin{columns}
        \begin{column}{0.5\textwidth}
            \includegraphics[width=\textwidth]{../figuras/consumo3.png}
        \end{column}
        \begin{column}{0.5\textwidth}
            \centering
            \textbf{\Large Quem compra com juros paga mais caro pelo mesmo produto.}
        \end{column}
    \end{columns}
\end{frame}

\begin{frame}[c]
    \frametitle{}
    \begin{columns}
        \begin{column}{0.5\textwidth}
            \centering
            \textbf{\Large Quem paga juros reduz seu poder de consumo.}
        \end{column}
        \begin{column}{0.5\textwidth}
            \includegraphics[width=\textwidth]{../figuras/consumo2.png}
        \end{column}
    \end{columns}
\end{frame}



\begin{frame}[c]\frametitle{Estratégia para quitar as dívidas}
    % \textbf{Minha Estratégia para quitar as dívidas}
    \begin{itemize}
        \item Ajustar os gastos ao ganho mensal.
        \item Usar o $13^\circ$ e adicional de férias para pagar dívida.
        \item Tentar gerar renda extra.
        \item Não aumentar os gastos devido a aumento salarial.
    \end{itemize}
\end{frame}

\begin{frame}[c]
    \frametitle{}
    \begin{columns}
        \begin{column}{0.5\textwidth}
            \includegraphics[width=\textwidth]{../figuras/aluguel2.png}
        \end{column}
        \begin{column}{0.5\textwidth}
            \centering
            \textbf{\Large Juros é o aluguel do dinheiro.}
        \end{column}
    \end{columns}
\end{frame}

\begin{frame}[c]\frametitle{Podemos alugar nosso dinheiro para}
    % \textbf{Minha Estratégia para quitar as dívidas}
    \begin{itemize}
        \item Bancos através de CDB, LCI, LCA, etc.
        \item Governo através dos títulos do Tesouro Direto.
        \item Empresas através de Debêntures.
    \end{itemize}
\end{frame}
