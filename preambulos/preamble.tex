% Preambulo com pacotes para o documento

% Pacotes de Matemática
\usepackage{amsmath}    % Melhora a formatação de fórmulas matemáticas
\usepackage{amsthm}     % Facilita a criação e formatação de teoremas
\usepackage{amsfonts}   % Fontes matemáticas adicionais
\usepackage{amssymb}    % Símbolos matemáticos adicionais
\usepackage{amscd}      % Diagramas de comutatividade
\usepackage{mathrsfs}   % Fonte cursiva para símbolos matemáticos
\usepackage{bbm}        % Fonte Blackboard Bold

% Pacotes de Algoritmos e Programação
\usepackage[ruled,lined]{algorithm2e}   % Para criar algoritmos e pseudocódigos

% Pacotes para Texto e Layout
\usepackage{caption}    % Personaliza legendas de figuras e tabelas
\usepackage{microtype}  % Melhora a tipografia
\usepackage{todonotes} % Adiciona notas e lembretes
\usepackage{lipsum}     % Gera texto de exemplo (Lorem Ipsum)
\usepackage{geometry}   % Configura margens e dimensões de página
\usepackage{ragged2e}   % Ajusta a justificação do texto
\usepackage{multicol}   % Divide o texto em várias colunas
\usepackage{pdfsync}    % Sincroniza PDF com código-fonte
% \usepackage{lmodern}    % Usa fontes Latin Modern

% Pacotes para Gráficos e Tabelas
\usepackage{graphicx}   % Inclui e manipula gráficos e imagens
\usepackage{float}      % Melhora o posicionamento de figuras e tabelas
\usepackage{subfig}     % Cria figuras compostas por subfiguras
\usepackage{tabularx}   % Tabelas com largura ajustável
\usepackage{booktabs}   % Melhora a aparência das tabelas com linhas horizontais

% Pacotes para Cores e Estilo
\usepackage{xcolor}    % Extensão do pacote color com mais opções
\usepackage{colortbl}  % Colorir tabelas

% Pacotes para Referências e Bibliografia
\usepackage{biblatex}  % Gerencia bibliografias e citações

% Pacotes para Listas e Enumerações
% \usepackage{enumitem}  % Controle sobre listas enumeradas e com marcadores

% --- Pacote para ÍCONES ---
\usepackage{fontawesome5}

% Outros Pacotes
\usepackage{siunitx}   % Formata unidades e números no Sistema Internacional de Unidades (SI)
\usepackage{url}       % Melhora a formatação e quebra de URLs longos
\usepackage{xspace}    % Gerencia o espaço após macros de texto

\synctex=1 % Habilita SyncTeX

%  Define as cores a serem usadas
\definecolor{UFGblue}{rgb}{0.0039, 0.3529, 0.6431}
\definecolor{UFGred}{HTML}{990000}
\definecolor{UFGorange}{rgb}{0.9216, 0.4863, 0.0784}
\definecolor{UFGgreen}{rgb}{0.0509, 0.4509, 0.1568}
\definecolor{UFGgray}{rgb}{0.3686, 0.5255, 0.6235} % UBC Gray (secondary)
\definecolor{ultramarine}{rgb}{0, 0.125, 0.376}
\definecolor{UFGorange}{rgb}{0.9216, 0.4863, 0.0784}%{HTML}{EB811B}
\definecolor{UFGred}{HTML}{990000}

\definecolor{mybcolor}{rgb}{0.122, 0.435, 0.698}
\definecolor{mygcolor}{rgb}{0.0, 0.7, 0.2}
\definecolor{myrcolor}{rgb}{0.8, 0.0, 0.2}

\definecolor{darkgreen}{rgb}{0.0509, 0.4509, 0.1568}
\definecolor{orangeb}{rgb}{0.9216, 0.4863, 0.0784}%{HTML}{EB811B}
\definecolor{blued}{rgb}{0.0039, 0.3529, 0.6431}

% Comandos para colorir o texto
\newcommand{\red}[1]{\textcolor{UFGred}{#1}}
\newcommand{\blue}[1]{\textcolor{UFGblue}{#1}}
\newcommand{\green}[1]{\textcolor{UFGgreen}{#1}}
\newcommand{\gray}[1]{\textcolor{UFGgray}{#1}}
\renewcommand{\leq}{\leqslant}
\renewcommand{\geq}{\geqslant}

% Criando novos comandos
\DeclareMathOperator{\arsinh}{arsinh}
\DeclareMathOperator{\corr}{Corr}
\DeclareMathOperator{\cov}{Cov}
\DeclareMathOperator{\var}{Var}

% definindo as margens do documento
\geometry{a4paper,text={16.5cm,25.2cm},centering}

% Definindo os conjuntos numéricos 
\newcommand{\C}{\mathbb{C}}
\newcommand{\N}{\mathbb{N}}
\newcommand{\Q}{\mathbb{Q}}
\newcommand{\R}{\mathbb{R}}
\newcommand{\Z}{\mathbb{Z}}


