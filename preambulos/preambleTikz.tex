% Pacotes para lógica condicional
\usepackage{ifthen} % Fornece suporte para instruções condicionais básicas
\usepackage{xifthen} % Fornece suporte avançado para lógica condicional

% Pacotes para gráficos e diagramas
\usepackage{pgf} % Pacote base para gráficos e diagramas com TikZ
\usepackage{pgfplots} % Extensão do PGF para gráficos científicos e matemáticos
\pgfplotsset{compat=1.16} % Define a compatibilidade para a versão 1.16
\usepgfplotslibrary{groupplots} % Carrega a biblioteca para gráficos agrupados
\usepackage{pgfkeys} % Gerenciamento de opções e chaves para PGF
\usepackage{tikz} % Criação de gráficos vetoriais e diagramas
% \usepackage{tkz-euclide} % Extensão do TikZ para criar figuras geométricas euclidianas

% Pacotes para funções matemáticas
\usepackage{tkz-fct} % Para desenhar gráficos de funções matemáticas

% Pacotes para árvores e grafos
\usepackage{forest} % Para criar diagramas de árvores e grafos

% Pacotes para unidades e formatação
\usepackage{siunitx} % Sistema Internacional de Unidades, formatação de unidades e números

% Pacotes para legendas e cores
\usepackage{xcolor} % Suporte para cores em gráficos e texto
\usepackage{graphicx} % Suporte para incluir gráficos e imagens
\usepackage{caption} % Personalização de legendas de figuras e tabelas

\usepackage{pgfplots}
% Carregamento das bibliotecas do TikZ em ordem alfabética
\usetikzlibrary{
  angles,              % Desenho e rotulação de ângulos
  automata,            % Desenho de autômatos e máquinas de estados
  arrows,              % Estilos de setas e flechas
  arrows.meta,         % Estilos avançados de setas
  backgrounds,         % Planos de fundo e camadas
  calc,                % Cálculos matemáticos em coordenadas
  decorations.markings,% Decorações e marcas em linhas
  fillbetween,         % Áreas preenchidas entre curvas
  intersections,       % Interseção de figuras geométricas
  math,                % Suporte a expressões matemáticas
  patterns,            % Padrões de preenchimento
  positioning,         % Posicionamento de nós e figuras
  quotes,              % Citações e anotações
  shapes.geometric,    % Formas geométricas básicas
  trees,               % Diagramas de árvores
  3d,                  % Representações tridimensionais
  fit,                 % Ajuste e posicionamento de figuras
  external             % Salvamento de gráficos como arquivos externos
}

% Define um novo comando para obter coordenadas em milímetros
\makeatletter % Permite usar o caractere @ em nomes de comandos
\newcommand{\gettikzxy}[3]{%
  \tikz@scan@one@point\pgfutil@firstofone#1\relax % Analisa o ponto fornecido
  \edef#2{\the\pgf@x}% Define a coordenada x em #2
  \edef#3{\the\pgf@y}% Define a coordenada y em #3
  \tikzmath{#2=#2/28.3465;}; % Converte coordenada x de pontos para milímetros
  \tikzmath{#3=#3/28.3465;}; % Converte coordenada y de pontos para milímetros
}
\makeatother % Restaura o comportamento padrão do caractere @

% Define um estilo para nós TikZ
\tikzstyle{bag} = [align=center] % Centraliza o conteúdo do nó


% to Bezier curves
\newcommand\DrawControl[3]{
node[#2,circle,fill=#2,inner sep=2pt,label={above:$#1$},label={[black]below:{\footnotesize#3}}] at #1 {}
}

