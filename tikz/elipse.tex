\begin{tikzpicture}[xscale=0.5]

    \coordinate (O) at (0,0);
    \coordinate (i) at (9,0);
    \coordinate (j) at (0,4);

    % Título do gráfico
    % \node[above=0.5cm] at (j) {Elipse}; % Título acima da letra y

    % Define a área visível do gráfico
    \path[draw=none] (-10,-2) rectangle (10,6.5); % Retângulo invisível

    % Desenha os eixos
    \draw[-latex] (-9,0) -- (i);
    \draw[-latex] (0,-2) -- (j);

    % Rótulo do eixo X
    \node[right] at (i) {$x$};

    % Rótulo do eixo Y 
    \node[above] at (j) {$y$};

    % Rótulo do ponto O
    \node[anchor=south east] at (O) {$O$};

    % Desenha a elipse
    \draw[thick,color=black,domain=-8:8,smooth, samples=500]
    plot (\x,{1 + sqrt(4-(\x)^2/16)});

    \draw[thick,color=black,domain=-8:8,smooth, samples=500]
    plot (\x,{1 - sqrt(4-(\x)^2/16)});

    % Linha horizontal do eixo da elipse
    \draw[dotted,color=magenta,domain=-8:7.8,smooth]
    plot (\x,{1});

    % Texto explicativo da equação
    \node at (5,3.5) {$\frac{x^2}{64}+\frac{(y-1)^2}{4}=1$};

    % Focos da elipse
    % \filldraw[black] (7.74,1) circle (1pt); % Foco direito
    % \filldraw[black] (-7.74,1) circle (1pt); % Foco esquerdo

    % Marca os vertices da elipse
    \filldraw[black] (8,1) circle (1pt); % Foco direito
    \filldraw[black] (-8,1) circle (1pt); % Foco esquerdo
    \filldraw[black] (0,3) circle (1pt); % Foco superior
    \filldraw[black] (0,-1) circle (1pt); % Foco inferior

\end{tikzpicture}
